\documentclass[a4paper,12pt,twoside]{article}%
\usepackage[T1]{fontenc}%
\usepackage[utf8]{inputenc}%
\usepackage{lmodern}%
\usepackage{textcomp}%
\usepackage{lastpage}%
\usepackage{geometry}%
\geometry{tmargin=20mm,bmargin=20mm,lmargin=25mm,rmargin=25mm}%
\usepackage{graphicx}%
\usepackage{ragged2e}%
\usepackage{amsmath}%
%
%
%
\begin{document}%
\normalsize%


\begin{figure}[h!]%
\centering%
\includegraphics[width=70px]{C:/Users/mateu/OneDrive/Área de Trabalho/ECP/2020.1/analise de sistemas lineares/Codigos/latex/imagens/ufma.png}%
\end{figure}

%
\begin{center}%
{\large \rm \textbf {Universidade Federal do Maranhão} \linebreak}%
{\large \rm \textbf {Coordenação do Curso de Engenharia da Computação} \linebreak}%
\end{center}%
\section{The fancy stuff}%
\label{sec:Thefancystuff}%
\subsection{Correct matrix equations}%
\label{subsec:Correctmatrixequations}%
\[%
\begin{pmatrix}%
2&3&4\\%
0&0&1\\%
0&0&2%
\end{pmatrix} \begin{pmatrix}%
100\\%
10\\%
20%
\end{pmatrix} = \begin{pmatrix}%
310\\%
20\\%
40%
\end{pmatrix}%
\]

%
\subsection{Alignat math environment}%
\label{subsec:Alignatmathenvironment}%
\begin{alignat*}{2}%
\frac{a}{b} &= 0 \\%
\begin{pmatrix}%
2&3&4\\%
0&0&1\\%
0&0&2%
\end{pmatrix}%
\begin{pmatrix}%
100\\%
10\\%
20%
\end{pmatrix}%
&=%
\begin{pmatrix}%
310\\%
20\\%
40%
\end{pmatrix}%
\end{alignat*}

%
$\int_{a}^{b} x^2 \,dx$%
\end{document}